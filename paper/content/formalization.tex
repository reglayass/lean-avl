I begin by formalizing a binary search tree. Lean already has a definition
for a binary tree, \lstinline{bin_tree}, but it does not fulfill the requirements for a search tree. 
I want to be able to have a key and a value for each tree node, and left and right child nodes (which may be empty).
Figure \ref{fig:btree_def} shows the new implementation of a binary tree.

\begin{figure}[!h]
  %\lstinputlisting[language=lean, firstline=3, lastline=5, frame=single]{/Users/sofiakonovalova/Desktop/Thesis/lean-thesis/src/basic.lean}
  \caption{}
  \label{fig:btree_def}
\end{figure}

\notes{Why using an inductive definition? Add lookup and bound definitions}

I am also formalizing the binary search property as described in Definition \ref{def:bst_property}, as the inductive definition above does not 
guarantee an ordered tree.
For a search tree to be ordered means that all the keys in its left subtree are smaller than the root, and all the keys in its right child subtree 
are larger than the root. This also has to hold for all the subtrees in the structure.

\begin{figure}[!h]
  %\lstinputlisting[language=lean, firstline=35, lastline=38, frame=single]{/Users/sofiakonovalova/Desktop/Thesis/lean-thesis/src/basic.lean}
  \caption{}
  \label{fig:forall_keys}
\end{figure}

\notes{explain what forall\_keys is}

\begin{figure}[!h]
  %\lstinputlisting[language=lean, firstline=40, lastline=42, frame=single]{/Users/sofiakonovalova/Desktop/Thesis/lean-thesis/src/basic.lean}
  \caption{}
  \label{fig:ordered}
\end{figure}
