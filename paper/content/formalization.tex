\section{Formalization}

\subsection{Simple BST}
The first step to formalizing an AVL tree was to formalize a simple binary search tree. Even though Lean has a definition of a basic binary tree 
\lstinline{bin_tree}, that definition does not include a node key. For a binary search tree, this is important, so the inductive type for this formalization includes 
both a key and a value for each node.

\begin{figure}[!ht]
  \lstinputlisting[language=lean, firstline=3, lastline=5, frame=single]{/Users/sofiakonovalova/Desktop/Thesis/lean-thesis/src/basic.lean}
  \caption{}
  \label{fig:btree_simple}
\end{figure}

Figure \ref{fig:btree_simple} shows the inductive type for the binary tree. This simple simple tree has two
constructors: an empty binary tree, and a node with a left and right subtree, a key and a node value.

Figure \ref{fig:lookup_simple} shows a simple lookup operation. As mentioned before, nodes in a binary search tree are ordered; all keys in the left child subtree are smaller than the root,
and all keys in the right child subtree are larger than the root. Therefore when looking up a key in the tree, we need to do a comparison. If the key that is being looked up is smaller, then do a recursive lookup
on the left subtree; if it is larger, then do a recursive lookup on the right subtree. If the key is neither larger nor smaller, than we have found the key.

The \lstinline{bound} is very similar to \lstinline{lookup}. The definition checks if a key exists in the tree.

\begin{figure}[!ht]
  \lstinputlisting[language=lean, firstline=12, lastline=17, frame=single]{/Users/sofiakonovalova/Desktop/Thesis/lean-thesis/src/basic.lean}
  \caption{}
  \label{fig:lookup_simple}
\end{figure}

\begin{figure}[!ht]
  \lstinputlisting[language=lean, firstline=19, lastline=24, frame=single]{/Users/sofiakonovalova/Desktop/Thesis/lean-thesis/src/basic.lean}
  \caption{}
  \label{fig:bound_simple}

The insertion operation also does key comparisons in order to pick which child subtree to enter the new node into.
\end{figure}