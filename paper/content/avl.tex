\section{AVL Trees}

\subsection{Binary Search Trees}
First we begin by defining what exactly a binary tree is. A binary tree is a tree data structure in which a node has at most two children. These children are called
the left child and the right child. We can define binary trees recursively using set notation: a non-empty binary tree is a tuple $(L, S, R)$ where $L$ and $R$ are the left and right
subtrees respectively, and $S$ is a singleton set containing only the root node \cite{tree:set_not}. A binary search tree is a special case of a binary tree. In a binary search tree, all keys 
in the left subtree are lesser than of a node, and all keys in the right subtree are greater than of a node. This is sometimes called the \textit{binary search property}. This allows for faster lookup and insertion, as the ordering means that
half of the tree can be skipped, so lookup and insertion takes in the worst case $\log_2 n$ time, with $n$ being the number of nodes in the tree.

Insertion and lookup can be done recursively or iteratively, where at each node a comparison is made between the new key and the key of the current node. If the new key is smaller, then we compare the left node. Otherwise, we compare with the right node.

AVL trees are built on a binary search trees. However, as AVL trees are \textit{self-balancing}, rotation algorithms need to be defined to be used during certain cases of insertion.

\subsection{AVL tree construction}
Since AVL trees are based on binary search trees, there is not much that needs to be done in terms of construction. However, as mentioned previously, AVL trees are \textit{self-balancing}. 
We begin by defining the \textit{balancing factor}, which defines what it means for a tree to be balanced in the first place.

\begin{definition}[Balancing factor] 
  The balancing factor $BF(N)$ of a node $N$ is defined as the height difference between its two subtrees. A tree is an AVL tree if the invariant
  $BF(N) \in \{-1, 0, 1\} $ for all nodes in the tree.
\end{definition}

In general, read-only operations on AVL trees function the same as in binary search trees. Modifications are made in the insertion operations, where 
a change in the balancing factor needs to be observed and taken care of with rotation algorithms.