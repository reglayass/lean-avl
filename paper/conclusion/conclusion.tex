Theorem provers such as Lean are used to create libraries of mathematics and logic, to be used for research and teaching. In this thesis, I presented a formalization of AVL search trees using Lean. Lean proof constructions a relatively simple and mechanical task, once the main hurdle of learning and getting used to the syntax. Overall, the proof statements presented in Lean were not very different to those presented informally in \cite{textbook:discrete_computer}, although some very small changes did have to be made. Additionally, definitions regarding the structure of AVL treees such as \lstinline{balance} and \lstinline{height} stay true to their mathematical definitions for trees. Hopefully, in the future, formalization of search trees as a form of contribution to mathematical libraries in Lean can continue and other abstract data structures may be formalized as well.

\subsection*{Further Work}
The continuation of this work would begin by completing proofs for insertion and deletion that are missing from the source code, for which the only cause is a lack of time. The next steps are to make use of Lean's automation and metaprogramming to create shorter and cleaner code, as well as improving definitions to remove overhead. The final step is the inclusion of this final formalization into the mathlib library.