Formalizations of AVL trees, as well as other search trees, are present in both the Coq \cite{code:coq} and Isabelle/HOL \cite{isabelle} interactive theorem provers. Lean does not have an AVL tree formalization yet, however other self-balancing trees are present in the mathlib library, such as red-black trees.

\subsection*{Coq}
In contrast to my approach, Coq has one single balance function, \lstinline{bal}, and no abstraction into separate left/right rotations. This can be seen as a stylistic choice, but it creates a very long definition and longer, more complicated proofs. While in my approach, the proofs are still relatively long, because lemmas are written for separate rotations, they can be applied to the long proof. It can have an effect on performance. The \lstinline{bal} function is used in insertion, which is recursive. My approach to the \lstinline{insert} function is to first determine whether it will result in a left- or right-heavy tree, and then rotate the tree after insertion, which saves operational time on not balancing a tree when it is not necessary.  

The definition of binary search trees in Coq and Lean are similar, the only difference is that height is included in the definition of trees in Coq, while in Lean height is calculated when needed. While this can be a stylistic choice, it may have an effect on performance. With every new node added, one is added to the height of each ancestor, while with the Lean interpretation height is calculated on demand, which takes longer the bigger the tree is.

\subsection*{Isabelle}
Unlike in Coq, in Isabelle there are two functions \lstinline{balL} that re-balances left subtrees and \lstinline{balR} that re-balances right subtrees. They function similarly to \lstinline{rotate_right} and \lstinline{rotate_left}, and are used in the balancing function. This is better in terms of proof construction as it makes the proof more modular. Like in Coq, the height of the tree is included in the tree definitions.

\subsection*{Other Search Trees}
Isabelle/HOL has a library of data structures, which apart from containing an AVL tree formalization, has other tree formalizations like red-black trees, 2-3 trees, and standard binary trees. In Coq, finite sets are implemented using trees and are encapsulated in the MSet libary. 

Lean itself has an inductive datatype \lstinline{bin_tree}, but no operations related to it. The mathlib library \cite{The_mathlib_Community_2020}, a standard library in Lean, has an inductive type \lstinline{tree} which is similar to \lstinline{btree}. Both of the definitions do not contain a key value in the type constructors, so as is, they cannot be used to create search trees, or create maps or sets. Red-black trees have definitions for the inductive type \lstinline{rbnode} and some definitions, but it is not completely formalized. 