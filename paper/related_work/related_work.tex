Formalizations of AVL trees, as well as other search trees, are present in both Coq \cite{code:coq} and Isabelle/HOL \cite{isabelle} interactive theorem provers. While not having an AVL tree formalization in Lean yet, self-balancing trees in the form of red-black trees are present.

\subsection*{Coq}
In terms of differences with regards to definitions, Coq has one single balance function, and no abstraction into separate left/right rotations. This can be seen as a stylistic choice, but it creates a very long definition and can create longer, more complicated proofs. It may also have an effect on performance, as the definition is pulling more weight of the operation. The balanced insert function re-balances using the \lstinline{bal} function after recursive insertion into the left or right tree, which may be inefficient as insertion is recursive as well as the \lstinline{bal} definition.

The definition of binary search trees in Coq and Lean are similar, the only difference is that height is included in the definition of trees in Coq, while in Lean height is calculated when needed. While this can be a stylistic choice, it may have an effect on performance. With every new node added, one is added to the height of the parent, while with the Lean interpretation height is calculated on demand, which takes longer the bigger the tree is.

\subsection*{Isabelle}
Unlike in Coq, in Isabelle there are two functions \lstinline{balL} that re-balances left subtrees and \lstinline{balR} that re-balances right subtrees. They function similarly to \lstinline{rotate_right} and \lstinline{rotate_left}, and is used in the balancing function. This is better in terms of proof construction than the single balance function present in Coq, and allows for proofs to be constructed with regards to the separate balancing definitions and then the whole balance definition, like in the Lean formalization. Height is included in the definition of binary search trees, just like in Coq. The balanced insert function is also the same as in Coq.

\subsection*{Other Search Trees}
Isabelle/HOL has a library of data structures, which apart from containing an AVL tree formalization, has other tree formalizations, like red-black trees, 2-3-trees, and of course standard trees. In Coq, sets are implemented using trees and therefore all tree structures are encapsulated into the library MSet for finite sets.

Lean itself has an inductive datatype \lstinline{bin_tree}, but no operations related to it. In the mathlib library, another inductive type \lstinline{tree} which is similar to \lstinline{bin_tree}. Both the of three definitions do not include key values. The one tree that is formalized is a red-black tree, with lemmas missing but operations being fully formalized. 