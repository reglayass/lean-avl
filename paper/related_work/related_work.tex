Formalizations of AVL trees are present in both Coq\cite{code:coq} and Isabelle/HOL\cite{isabelle} interactive theorem provers. This section will compare the formalizations present in the these languages and my implementation in Lean.

First, some differences. Both the Isabelle and Coq implementations of balancing a tree are different to the implementation in Lean. In Isabelle, there are two functions \lstinline{balL} that re-balances left subtrees and \lstinline{balR} that re-balances right subtrees. Coq has one single balance function, and no abstraction into separate left/right rotations. The functions in Isabelle functional essentially the same as \lstinline{rotate_right} and \lstinline{rotate_left} but don't abstract the single rotations away. This can be in part just a difference of stylistic choice, but in the case of having one single balance function, it can cause efficiency issues, and create longer proof constructions. Both Coq and Isabelle have the height of a tree starting from a node in the node data, while in Lean the height is calculated when needed using a function. This is purely a stylistic choice. 

Now the similarities. Coq has a function \lstinline{remove_min} and Isabelle \lstinline{split_max} which act in a similar way to \lstinline{shrink} - splitting off a key and a node from a tree, and performing operations on the subtree left. The implementations of the binary search tree base is also the same for the exception of height being part of the binary tree structure. Coq goes on to later use a binary search tree as part of a different inductive type \lstinline{avl}, with assumptions on height.