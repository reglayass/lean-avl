In computer science, interactive theorem provers are used to develop formal proofs of certain mathematical or computer science concepts. The advantage to using a theorem prover to writing a proof by hand is that these proofs could either be long or difficult to prove by hand. With theorem provers, every step of proof construction is verified, which gives confidence in the correctness of the proof itself and the proof statement. Interactive theorem provers are also used by mathematics researchers to build up a repository of formalized mathematics. 

One of the ways that Lean has been built up with over time is with the mathlib library \cite{The_mathlib_Community_2020}. The library has now become a defacto standard library in Lean, but started as a community effort in order to formalize mathematical concepts in Lean. The mathlib library also includes binary trees and red-black trees. The goal of this paper is to fully formalize the tree main components of AVL tree rotations -- order, key preservation and balance -- and ultimately integrate a polished version of the resulting definitions and proofs into the mathlib library.

First, this thesis gives an introduction on binary search trees and AVL search trees and their operations, providing the corresponding Lean definitions and types. The thesis then discusses the verification process, giving some important lemmas in relation to AVL tree operations and mentions design changes that were made along the way. Finally, the formalization is compared to those made in other interactive theorem provers and the future of this work is outlined.