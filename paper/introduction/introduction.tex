In computer science, interactive theorem provers are used to develop formalizations of mathematical or logical concepts. The advantage of using these provers is that proofs can be long, or difficult to write by hand -- a good example is Fermat's Last Theorem, the full proof of which is known to be thousands of pages long. Additionally, there is no guarantee of the correctness of the initial proof statement. With interactive theorem provers, every proof step is verified by the lnaguage, which gives confidence in the correctness of the proof itself and the proof statement.

Interactive theorem provers, such as Coq, Isabelle and Lean, are used to develop formalizations of mathematical and logical concepts. Lean, though it is younger than Coq or Isabelle, already boasts mathlib \cite{The_mathlib_Community_2020}, a large library of formalized mathematics. Lean also improves on Coq by having a smaller kernel, cleaner syntax with Unicode support that can be written in Lean source files, and support for metaprogramming. 

Though matlib, through its detail and extensiveness has become a defacto standard library in Lean, is missing something, contrary to its counterparts Coq and Isabelle -- formalized tree structures. This thesis serves as a contribution to Lean and the mathlib library in the form of formalizing AVL trees, as well as a case study of the possibility of further formalization of search trees in Lean. 

The paper begins by giving a brief introduction to the Lean theorem prover, continuing on to a brief introduction on binary search trees and AVL trees, providing the corresponding Lean types and definitions. Afterwards, the verification process is discussed, proving all the necessary proof statements and changes made along the way. Finally, the formalization is compared to those made in Coq and Isabelle, and the future of this work is outlined.