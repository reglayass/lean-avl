During most proof constructions presented in this section, I simplified definitions to extract information, or create case splits. With \lstinline{shrink}, this process would be long and would clutter the proof. There needed to be a way to split the result of \lstinline{shrink} into the tree possible cases that can come out of shrinking a tree with one action, reducing any unnecessary writing. The same problem arose with \lstinline{del_node}. To solve these problems we define two views for \lstinline{shrink}\footnote{This approach was suggested by Jannis Limperg} and \lstinline{del_node}. The \lstinline{shrink_view} is shown below.

\begin{lstlisting}
inductive shrink_view {α} : btree α → option (nat × α × btree α) → Sort*
| empty : shrink_view empty none
| nonempty_empty : ∀ {l k v r},
  shrink r = none →
  shrink_view (node l k v r) (some (k, v, l))
| nonempty_nonempty₁ : ∀ {l k v r x a sh out},
  shrink r = some (x, a, sh) →
  height l > height sh + 1 →
  out = some (x, a, rotate_right (btree.node l k v sh)) →
  shrink_view (node l k v r) out
| nonempty_nonempty₂ : ∀ {l k v r x a sh},
  shrink r = some (x, a, sh) →
  height l ≤ height sh + 1 →
  shrink_view (node l k v r) (some (x, a, node l k v sh))
\end{lstlisting}

The views have a constructor for each possible result of \lstinline{shrink} or \lstinline{del_node}. In the case where rotations are made, an adjustment had to be made in the form of the assumption \lstinline{out}. This was done because inductive types in Lean do not accept other function calls in the type constructor.

In order to use the views, auxiliary lemmas were written to apply a normal \lstinline{shrink} or \lstinline{del_node} and get the three case splits. The lemma for \lstinline{shrink_view} is shown below. The proof for \lstinline{del_node} is almost identical.

\begin{lstlisting}
lemma shrink_shrink_view (t : btree α) : 
  shrink_view t (shrink t) := ...
\end{lstlisting}

Applying the lemma would result in three cases in the inductive step of \lstinline{shrink_ordered}, which matches the three cases from the definition -- one for the right subtree being empty, leading to \lstinline{shrink r = none}, another one where \lstinline{shrink r = some (x, a, rotate_right(l k v sh))}, and another one where \lstinline{shrink r = some (x, a, node l k v sh)}. This allowed to complete the proofs without creating additional case splits, cluttering the proof construction.