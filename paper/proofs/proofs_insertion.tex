For insertion, we want to show that keys, order and balance are preserved.

\subsection*{Key Preservation}
There are three different lemmas in relation to key preservation.

\begin{lstlisting}
lemma insert_bound (t : btree α) (k : nat) (v : α) :
  bound k (insert k v t) := ...

lemma insert_diff_bound (t : btree α) (k x : nat) (v : α) :
  bound x t → bound x (insert k v t) := ...

lemma insert_nbound (t : btree α) (k x : nat) (v : α) :
  (bound x t = ff ∧ x ≠ k) → bound x (insert k v t) = ff := ...
\end{lstlisting}

For \lstinline{insert_bound}, the goal is to show that a key can be found as soon as it is inserted into a tree. For \lstinline{insert_diff_bound} and \lstinline{insert_nbound}, the goal is to show that the keys already existing in a tree do not get lost during insertion, and if a key does not exist in a tree, unless it is the new key, it does not exist after insertion. The proofs are completed by induction on the tree \lstinline{t}.

Since \lstinline{insert} uses rotations, previously constructed proofs about rotations preserving keys are applied in these proofs. The tactic \lstinline{tauto} is used frequently, as the tactic is a decisional procedure for propositional logic and is able to split goals and assumptions that are disjunctions. Due to the \lstinline{bound} definition using disjunction, this tactic does a lot of the heavy work of breaking down these forms, and then completing the separate foals based on reflexivity.

\subsection*{Order}
Lemma statements for insertion preserving order have the same structure as lemmas for rotations preserving order.

\begin{lstlisting}
lemma insert_ordered (t : btree α) (k : nat) (v : α) :
  ordered t → ordered (insert k v t) := ...
\end{lstlisting}

This proof is constructed by induction on the tree \lstinline{t} and previous lemmas for rotations preserving order are applied. 

To complete the proof, an auxiliary proof \lstinline{forall_insert} is written. In the process of completing \lstinline{insert_ordered}, we need to show that previously existing keys in a tree preserve their relation with the tree after insertion. To bring an example from the proof -- if a key \lstinline{tk} is greater than all of the keys in the tree \lstinline{tl} even after insertion, then it follows that \lstinline{tk > k} and that \lstinline{tk} is greater than all of the keys in \lstinline{tl}. These are also the sub-goals that come out after applying the lemma. 

\begin{lstlisting}
lemma forall_insert (k x : nat) (t : btree α) (a : α) 
(p : nat → nat → Prop) (h : p x k) :
  forall_keys p x t → forall_keys p x (insert k a t) := ...
\end{lstlisting}

Auxiliary proofs for rotation and \lstinline{forall_keys} are also written since insertion uses rotations.

\begin{lstlisting}
lemma forall_rotate_right (x k : nat) (l r : btree α) (a : α) 
(p : nat → nat → Prop) :
  forall_keys p x (btree.node l k a r) → 
    forall_keys p x (rotate_right (btree.node l k a r)) := ...

lemma forall_rotate_right (x k : nat) (l r : btree α) (a : α) 
(p : nat → nat → Prop) :
  forall_keys p x (btree.node l k a r) → 
    forall_keys p x (simple_right (btree.node l k a r)) := ...
\end{lstlisting}

The three proofs above were constructed by induction on \lstinline{t}.

\subsection*{Balance}
The lemma for insertion preserving balanced is formalized as such, and completed by induction on the tree \lstinline{t}.

\begin{lstlisting}
lemma insert_balanced (t : btree α) (k : nat) (v : α) :
  balanced t = tt → balanced (insert k v t) = tt :=
\end{lstlisting}

In the process of completing the proof, small changes were made to the \lstinline{insert} definition. Previously, the definition follow \cite{textbook:discrete_computer} almost exactly. In that case, the case splits would be made on the height differences. For example, to determine if a right rotation needs to be done after insertion, instead of using the definition \lstinline{left_heavy}, the comparison that was done was \lstinline{height (insert l) > height r} Since the proof for \lstinline{insert_balanced} would be composed of the proofs for rotation preserving balance, applying these lemmas would result in either a \lstinline{left_heavy} or \lstinline{right_heavy} goal, but the hypotheses would all be expressions with \lstinline{height}, which makes the proof more difficult to complete than what it should be. Therefore, in the definition, the cases that determine whether or not a rotation was done was changed from manual height comparisons to the \lstinline{left_heavy} and \lstinline{right_heavy} definitions. This meant that during the proof construction, there would be a case split on heaviness, and after applying a rotation lemma the goal would become trivial. 