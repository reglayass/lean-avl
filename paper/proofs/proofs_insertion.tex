Proofs about insertion into AVL trees, like the ones for rotation, need to show a preservation of order, keys, and balance restoration. This section will detail a proof about key preservation. 

\begin{lstlisting}[caption=\empty, label={lst:insert_1}]
lemma insert_balanced_diff_bound (t : btree α) (k x : nat) (v : α) :
  bound x t → bound x (insert_balanced k v t) := ...
\end{lstlisting}

Listing \ref{lst:insert_1} formalizes in Lean the proof of insertion not losing keys -- if a key is present in the tree before insertion, it will still be present after the insertion of a different key.

This proof construction makes use of two other lemmas, which state that rotation preserves keys. Since \lstinline{insert_balanced} uses rotations, these lemmas are applied in the main proof.

\begin{lstlisting}[caption=\empty, label={lst:rotate_keys}]
lemma rotate_left_keys (t : btree α) (k : nat) (x : bool) :
  bound k t = x → bound k (rotate_left t) = x := ...

lemma rotate_right_keys (t : btree α) (k : nat) (x : bool) :
  bound k t = x → bound k (rotate_left t) = x :=
\end{lstlisting}

The proof constructions for Listings \ref{lst:insert_1} and \ref{lst:rotate_keys} are completed with induction on \lstinline{t}, and are relatively simple from there, as they only consist of case splitting based on \lstinline{insert}, applying the rotation lemmas when needed and the goal can then be derived from the hypotheses with natural deduction rules. 