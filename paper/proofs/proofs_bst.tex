Since AVL trees are based on BSTs, and the \lstinline{lookup} and \lstinline{bound} definitions can be used on AVL trees without any additional rotations unlike with insertion, some proofs about these two operations were constructed.

\begin{lstlisting}
lemma bound_false (k : nat) (t : btree α) :
  bound k t = ff → lookup k t = none := ...

lemma bound_lookup (t : btree α) (k : nat) :
  bound k t → ∃ (v : α), lookup k t = some v := ...
\end{lstlisting}

The two definitions were used together in these proofs, as they are linked to each other by their purpose. If a key is not bound in a tree, then lookup will not result in any node data being returned. If a key is bound in a tree, then some data will be returned.