First, we prove two lemmas related to boundedness and lookup in a tree, since \lstinline{lookup} and \lstinline{bound} are identical for both BSTs and AVL trees.

\begin{lstlisting}
lemma bound_false (k : nat) (t : btree α) :
  bound k t = ff → lookup k t = none := ...

lemma bound_lookup (k : nat) (t : btree α) :
  bound k t → ∃ (v : α), lookup k t = some v := ...
\end{lstlisting}

The lemma \lstinline{bound_false} states that if a key is not bound in a tree, then lookup will not result in any node data being returned. The lemma \lstinline{bound_lookup} states that if a key is bound in a tree, then some data will be returned. The existential quantifier is used in this lemma, because we cannot make assumptions on which key will return which value. In other words, we don't know the specific node value, but we do know that something will be returned. Both of the proofs were constructed by induction on the tree \lstinline{t}.