All lemmas based on rotations preserving order follow the same structure for the statement: assuming that a tree \lstinline{t} is ordered, \lstinline{t} remainds ordered after a rotation. Since AVL trees are based on binary search trees, and binary search trees must be ordered, a rotation cannot violate this. 

\begin{lstlisting}
lemma rotate_right_ordered (t : btree α) :
  ordered t → ordered (rotate_right t) := ...
\end{lstlisting}

Since \lstinline{rotate_right} and \lstinline{rotate_left} are compound rotations, lemmas for simple rotations must be constructed too. 

\begin{lstlisting}
lemma simple_right_ordered (t : btree α) :
  ordered t → ordered (simple_right t) := ...

lemma simple_left_ordered (t : btree α) :
  ordered t → ordered (simple_left t) := ...
\end{lstlisting}

All of the above proofs were done by case splitting on the tree \lstinline{t}. Induction hypotheses wasn't needed since the definition for \lstinline{ordered} already splits on the left and right subtree, and the goal of the proof is to match the \lstinline{ordered} and \lstinline{forall_keys} subtrees of the hypothesis to the goal. 

To achieve this, a lemma for transitivity of keys is required, and this lemma is applied in both the lemmas for compound rotations and simple rotations.

\begin{lstlisting}
lemma forall_keys_trans (t : btree α) (p : nat → nat → Prop) 
(z x : nat) (h₁ : p x z) (h₂ : ∀ a b c, p a b → p b c → p a c) :
  forall_keys p z t → forall_keys p x t := ...
\end{lstlisting}

For example, if we have a key \lstinline{trk} and \lstinline{tk}, and the goal is to show that \lstinline{trk} is greater than all of the keys in a tree \lstinline{tl}, applying the lemma will yield three goals: one for \lstinline{trk > tk}, transivity (which is completed by a simple tactic \lstinline{apply trans}) and a goal \lstinline{forall_keys tk tl}. The two goals for the exception of transitivity should be available in the proof hypotheses. 