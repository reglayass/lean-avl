The process for constructing lemmas for balance is slightly different to the ones for ordering. We want show that balance restores rotation, therefore the assumption needs to be that a tree is either left or right imbalanced, and the corresponding rotation will make the tree balanced.

\begin{lstlisting}
lemma rotate_right_balanced (t : btree α) :
  left_heavy t → balanced (rotate_right t) := ...
\end{lstlisting}

As with ordering, the corresponding lemmas for simple rotations need to be constructed.

\begin{lstlisting}
lemma simple_right_balanced (t : btree α) :
  left_heavy t → balanced (simple_right t) := ...
  
lemma simple_left_balanced (t : btree α) :
  right_heavy t → balanced (simple_left t) := ...
\end{lstlisting}

Similarly to proofs about ordering, the above proofs were completed with case splitting on trees and subtrees. During construction of proofs with compound rotations, the lemmas about simple rotations were applied when needed. 

The proofs about balance were a bit more complicated to solve. With the ordering lemmas, it was a case of finding the goals in the hypotheses, with additionally using transitivity to achieve out goal. Since tree heaviness and the definition of \lstinline{balanced} use tree height, there was more mathematics involved. 