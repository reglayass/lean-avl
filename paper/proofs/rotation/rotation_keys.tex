The lemma for rotations preserving keys is written as so.

\begin{lstlisting}
lemma rotate_right_keys (t : btree α) (k : nat) (x : bool) :
  bound k t = → bound k (rotate_right t) := ...
\end{lstlisting}

The first step with proof statements with rotations is to look at their definitions. The definition for \lstinline{rotate_right} uses the \lstinline{simple_right} and \lstinline{simple_left} definitions, so proofs about them preserving order need to be constructed too.

\begin{lstlisting}
lemma simple_right_keys (t : btree α) (k : nat) (x : bool) :
  bound k t = → bound k (simple_right t) := ...
  
lemma simple_left_keys (t : btree α) (k : nat) (x : bool) :
  bound k t = → bound k (simple_left t) := ...
\end{lstlisting}

It was during the construction of these proofs where the first problem with the definition of \lstinline{bound} was discovered. Previously, the definition for \lstinline{bound} was similar to that of \lstinline{lookup}, recursively searching the tree until the key was found, returning true.

This definition would make it very difficult to write the proofs needed. There would be an assumption that the key is bound in the same subtree after a rotation, which is not the case after a rotation. The definition was then changed to the one shown in Section \ref{sec:bst}, which made the proofs easier because then the definition didn't take into account the placement of the key, just whether or not it exists. This fits better into the purpose of the proofs for rotation retaining keys - we want to make sure that some key is still present in the tree after a rotation, and not that the key is present in the same area of the tree.