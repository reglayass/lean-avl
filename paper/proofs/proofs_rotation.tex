With rotations we want to prove that an ordered tree preserves order after a rotation, and that if a tree is imbalanced then a rotation restores its balance. This section will present two proofs on right rotations preserving order and balance.

\subsection*{Order}
\begin{lstlisting}[caption=\empty, label={lst:right_ordered}]
lemma rotate_right_ordered (t : btree α) :
  ordered t → ordered (rotate_right t) := ...
\end{lstlisting}

Listing \ref{lst:right_ordered} shows a formalized lemma statement for right rotations preserving order. The first step with proof statements with rotations is to look at their definitions. The definition for \lstinline{rotate_right} uses the \lstinline{simple_right} and \lstinline{simple_left} definitions, so proofs about them preserving order need to be constructed too.

\begin{lstlisting}[caption=\empty, label={lst:simple_ordered}]
lemma simple_right_ordered (t : btree α) :
  ordered t → ordered (simple_right t) := ...

lemma simple_left_ordered (t : btree α) :
  ordered t → ordered (simple_left t) := ...
\end{lstlisting}

The proof in Listing \ref{lst:right_ordered} were done by case splitting on the tree \lstinline{t}, its right subtree \lstinline{r} and the next subtree \lstinline{lr}, and the simple rotation lemmas were applied throughout when needed. The simple rotation lemmas were also completed by case splitting on left or right subtree depending on the rotation.

The proofs above also require a lemma for transitivity of keys in trees. Assume a tree \lstinline{rr} and \lstinline{rl}, with the former having a left and a right child \lstinline{rll} and \lstinline{rlr}. Also assume two keys \lstinline{rk}, which is the parent node key of \lstinline{rr}, and \lstinline{rlk}, which is the key of \lstinline{rl}.
If \lstinline{rk} is greater than all the keys contained in \lstinline{rll} and \lstinline{rlr}, and \lstinline{rk} is less than the keys in \lstinline{rr}, by transitivity \lstinline{rlk} is less than all the keys in \lstinline{rr}. The lemma for key transitivity is formalized below.

\begin{lstlisting}[caption=\empty]
lemma forall_keys_trans (t : btree α) (p : nat → nat → Prop) 
(z x : nat) (h₁ : p x z) (h₂ : ∀ a b c, p a b → p b c → p a c) :
  forall_keys p z t → forall_keys p x t := ...
\end{lstlisting}

Another lemma had to be made to make constructing proofs with \lstinline{forall_keys} much easier. The current definition for \lstinline{forall_keys} does not take into consideration the relationship of the input key between the left and right children of the tree. The \lstinline{forall_keys_intro} introduction lemma solved this problem.

\begin{lstlisting}[caption=\empty]
lemma forall_keys_intro {l r : btree α} {k x : nat} {v : α} 
  {p : nat → nat → Prop} :
(forall_keys p k l ∧ p k x ∧ forall_keys p k r) 
  → forall_keys p k (node l x v r) := ...
\end{lstlisting}

When applying the introduction lemma, I was able to get the relation of the key between the left and right subtree, and the relation between the input key and the key of the tree, which made applying the transitivity lemma a lot easier.

\subsection*{Balance}